\documentclass[xcolor={table}]{beamer}

% Packages
\usepackage[brazil]{babel}	
\usepackage[utf8]{inputenc}
\usepackage[T1]{fontenc}
\usepackage[scaled]{helvet}
\usepackage{amsthm}
\usepackage{ragged2e}
\usepackage{subfig}
\usepackage[table]{xcolor}
\usepackage{multicol}
\usepackage{multirow}
\usepackage{fancyvrb}
\usepackage{verbatim}
\usepackage{minted}

% Configuration for packages
\usemintedstyle{bw}

% Theme
\usetheme{Execushares}

% Title page configuration
\title{Laborator IV: Mecanisme de Protecție}
\subtitle{}
\author{Iosif George-Andrei}
\setcounter{showSlideNumbers}{1}

\begin{document}

    % Title page
    \setcounter{showProgressBar}{0}
	\setcounter{showSlideNumbers}{0}
	\frame{\titlepage}

    % Table of content
	\begin{frame}
		\frametitle{Tabelă de Conținut}\pause
		\begin{enumerate}[<+->]
			\item Mecanisme de Protecție
			    \begin{enumerate}
    			\item Eliminarea Informațiilor
    			\item Împachetare
    			\item Canarii
    			\item Address Space Layout Randomization
    			\item Bitul NX
			\end{enumerate}
			\item Exerciții
		\end{enumerate}
	\end{frame}

    % First section
	\setcounter{framenumber}{0}
	\setcounter{showProgressBar}{1}
	\setcounter{showSlideNumbers}{1}
	\section{Mecanisme de Protecție}

	\begin{frame}
		\frametitle{Eliminarea Informațiilor I}\pause
		\begin{itemize}[<+->]
		    \item Execuția nu necesită toate informațiile existente într-un executabil.
		        \begin{itemize}
			        \item Numele unor simboluri (din secțiunile \mintinline{bash}{.symtab} și \mintinline{bash}{.dynsym})
			        \item Informații pentru depanare (din secțiunile specifice formatului DWARF, numite \mintinline{bash}{.debug_*})
			    \end{itemize}
        \end{itemize}
    \end{frame}

    \begin{frame}
		\frametitle{Eliminarea Informațiilor II}\pause
	    \begin{itemize}[<+->]
	        \item Măsură de securitate aplicată după compilarea executabilului și înainte de distribuirea lui către utilizatori
			\item Avantaje
			    \begin{itemize}
			        \item Reducerea dimensiunii executabilului
			        \item Execuție mai rapidă
			        \item Dezvăluirea a cât mai puține informații către utilizatori (eventual și atacatori)
			    \end{itemize}
		\end{itemize}
	\end{frame}
	
	\begin{frame}
		\frametitle{Eliminarea Informațiilor III}\pause
	    \begin{itemize}[<+->]
			\item Instrumente pentru eliminarea informațiilor
			    \begin{itemize}
			        \item \mintinline{bash}{strip}
			        \item \mintinline{bash}{gcc -s}
			    \end{itemize}
		\end{itemize}
	\end{frame}
	
	\begin{frame}
		\frametitle{Împachetare I}\pause
		\begin{itemize}[<+->]
		    \item Mecanism care comprimă executabilul curent, încorporând rezultatul într-un alt executabil (care se ocupă numai de despachetare)
		    \item Despachetare în memorie sau în fișier temporar
		    \item Ultimate Packer for eXecutables (abreviat UPX) ca cel mai cunoscut utilitar multi-platformă pentru împachetare
        \end{itemize}
    \end{frame}
    
    \begin{frame}
		\frametitle{Împachetare II}\pause
		\begin{itemize}[<+->]
		    \item Roluri
		        \begin{itemize}
			        \item Reducerea dimensiunii executabilului și îngreunarea analizei de către posibili atacatori
			        \item Reducerea dimensiunii programelor malițioase și îngreunarea analizei de către analiștii de securitate
			    \end{itemize}
        \end{itemize}
    \end{frame}
    
    \begin{frame}
		\frametitle{Canarii I}\pause
		\begin{itemize}[<+->]
		    \item Mecanismul constă în plasarea unor valori pe stivă, pentru a detecta tentativele de suprascriere.
		    \item Nume provenit de la păsările care intrau înaintea minerilor în subteran, pentru a detecta niveluri prea mari de gaz
        \end{itemize}
    \end{frame}
    
    \begin{frame}
		\frametitle{Canarii II}\pause
		\begin{itemize}[<+->]
		    \item Valori folosite
		        \begin{itemize}
			        \item Fixe, de obicei un terminator pentru posibile funcții de copiere a șirurilor de caractere
			        \item Aleatorii, de exemplu din \mintinline{bash}{/dev/urandom}
			        \item Altele, de exemplu rezultate ale unor \mintinline{bash}{xor}-uri
			    \end{itemize}
        \end{itemize}
    \end{frame}
    
    \begin{frame}
		\frametitle{Canarii III}\pause
		\begin{itemize}[<+->]
		    \item Tehnici de evaziune
		        \begin{itemize}
			        \item Deducerea valorii de canar
			        \item Folosirea de atacuri care pot scrie la o anumită zonă de memorie (de exemplu, cele cu șiruri de caractere de formatare)
			    \end{itemize}
        \end{itemize}
    \end{frame}
    
    \begin{frame}
		\frametitle{Address Space Layout Randomization I}\pause
		\begin{itemize}[<+->]
		    \item Constă în maparea segmentelor executabilului la adrese aleatorii de memorie.
		    \item Verificarea activării prin citirea conținutului \mintinline{bash}{/proc/sys/kernel/randomize_va_space}
        \end{itemize}
    \end{frame}
    
    \begin{frame}
		\frametitle{Address Space Layout Randomization II}\pause
		\begin{itemize}[<+->]
		    \item Segmente vizate
		        \begin{itemize}
			        \item Stivă
			        \item Librării dinamice (cu ajutorul secțiunilor \mintinline{bash}{.plt} și \mintinline{bash}{.got})
			        \item \textit{Heap}
			        \item Cod (numai la activarea mecanismului Position Independent Code)
			    \end{itemize}
        \end{itemize}
    \end{frame}
    
    \begin{frame}
		\frametitle{Address Space Layout Randomization III}\pause
		\begin{itemize}[<+->]
		    \item Tehnici de evaziune
		        \begin{itemize}
			        \item Atacuri cu forță brută
			        \item \mintinline{bash}{nop} \textit{sled}
			        \item \mintinline{bash}{jmp esp} sau \mintinline{bash}{call esp}
			        \item Obținerea unor informații despre memoria procesului
			    \end{itemize}
        \end{itemize}
    \end{frame}
    
    \begin{frame}
		\frametitle{Bitul NX}\pause
		\begin{itemize}[<+->]
		    \item Imposibilitatea unei pagini de a avea drepturi de scriere și execuție în același timp
		    \item Tehnici de evaziune
		        \begin{itemize}
			        \item Atacuri de tip Return Oriented Programming (abreviat ROP)
			        \item Apeluri către \mintinline{bash}{mprotect}
			        \item Atacuri de tip Return-to-libc
			    \end{itemize}
        \end{itemize}
    \end{frame}

	% Second section
	\section{Exerciții}
	
	\begin{frame}
		\frametitle{Exerciții}\pause
		\begin{enumerate}[<+->]
		    \item Verificarea Activării unor Mecanisme de Securitate
	    \end{enumerate}
	\end{frame}

	\begin{frame}
		\frametitle{Recomandări}\pause
		\begin{itemize}[<+->]
		    \item Folosiți comanda \mintinline{bash}{man} pentru a primi ajutor la rularea anumitor comenzi.
	    \end{itemize}
	\end{frame}

\end{document}